%--- synctex
\synctex=1
% \pdfoutput=1  % for arxiv submission

%--- math better math font
\usepackage{amsthm,amsmath,bbm,amsfonts,amssymb}
\usepackage{dsfont} % require for using \mathds{1}
\usepackage{mathtools}
\usepackage{commath}
\mathtoolsset{showonlyrefs=false}
\allowdisplaybreaks % This allows breaking multi-line equations

%--- font and spacing
\usepackage{xspace}
\usepackage[T1]{fontenc}
\usepackage[utf8]{inputenc}
\usepackage[usenames,dvipsnames]{xcolor} % for color names

%--- links
\usepackage{hyperref,url}
\usepackage{cleveref} % must be loaded after amsthm

%--- floating objects
\usepackage{wrapfig}
\usepackage[export]{adjustbox}
\def\imagetop#1{\vtop{\null\hbox{#1}}} % adjusting location of box in table
\usepackage{graphicx,tabularx}
\usepackage{multirow,hhline}% http://ctan.org/pkg/multirow
\usepackage{booktabs}% to use \midrule for drawing a horizontal line in align environment
\usepackage{pbox} % new line in a table cell -- http://tex.stackexchange.com/questions/2441/how-to-add-a-forced-line-break-inside-a-table-cell
\usepackage{algorithm,algorithmic}

%- for vertical alignment of includegraphics
%- http://tex.stackexchange.com/questions/101858/make-two-figures-aligned-at-top
\usepackage[export]{adjustbox}

%--- etc
\usepackage{capt-of}
\newcommand{\fix}{\marginpar{FIX}}
\newcommand{\new}{\marginpar{NEW}}

\usepackage{enumitem}


% define your comment color here
\newcommand{\kwang}[1]{{ \color[rgb]{1,0.279,0.998} #1 }}
\newcommand{\kw}[1]{{ \color[rgb]{1,0.279,0.998} #1 }}
% \newcommand{\kj}[1]{{\color{RedOrange}[#1]}}
\newcommand{\fo}[1]{{\color{Violet}[FO: #1]}}
\newcommand{\my}[1]{{\color{ForestGreen}[FO: #1]}}
\newcommand{\gray}[1]{{ \color[rgb]{.6,.6,.6} #1 }}
\newcommand{\blue}[1]{{\color[rgb]{.3,.5,1}#1}}
\newcommand{\tblue}[1]{{\color[rgb]{0,0,1}#1}} % 'the' blue
\newcommand{\grn}[1]{{\color{ForestGreen}#1}}
\newcommand{\org}[1]{{\color{Orange}#1}}
\newcommand{\vio}[1]{{\color{Violet}#1}}
\newcommand{\red}[1]{{\color[rgb]{1,.1,.1}#1}}
\newcommand{\gr}[1]{{\color[rgb]{.8,.8,.8}#1}}

\newcommand{\clb}[1]{{\color{blue}#1}}
\newcommand{\clg}[1]{{\color{green}#1}}
\newcommand{\clr}[1]{{\color{red}#1}}
\newcommand{\clc}[1]{{\color{cyan}#1}}

%---------- setup theorem environment
\usepackage{framed}
\usepackage[most]{tcolorbox}
\definecolor{kjgray}{rgb}{.7,.7,.7}

\newtheoremstyle{kjstyle}
{1ex} % Space above
{\topsep} % Space below
{\itshape} % Body font
{} % Indent amount
{\bfseries} % Theorem head font
{.} % Punctuation after theorem head
{.5em} % Space after theorem head
{} % Theorem head spec (can be left empty, meaning `normal')

\newtheoremstyle{kjstyle2}
{.0em} % Space above
{.0em} % Space below
{\itshape} % Body font
{} % Indent amount
{\bfseries} % Theorem head font
{.} % Punctuation after theorem head
{.5em} % Space after theorem head
{} % Theorem head spec (can be left empty, meaning `normal')


\newtheoremstyle{kjstylenoitalic}
{1ex} % Space above
{\topsep} % Space below
{} % Body font
{} % Indent amount
{\bfseries} % Theorem head font
{.} % Punctuation after theorem head
{.5em} % Space after theorem head
{} % Theorem head spec (can be left empty, meaning `normal')

\tcbset{kjboxstyle/.style={title={},breakable,colback=white,enhanced jigsaw,boxrule=1.3pt,sharp corners,colframe=kjgray,boxsep=0pt,coltitle={black},attach title to upper={},left=.8ex,bottom=.4em}}    

\tcbset{kjboxstylec/.style={title={},breakable,colback=white,enhanced jigsaw,boxrule=1.3pt,sharp corners,colframe=kjgray,boxsep=0pt,coltitle={black},attach title to upper={},left=.8ex,bottom=.4em,top=0.4em,enlarge top by=-0.3em,enlarge bottom by=-0.3em}}    
%    ,title after break={Lemma \thelemma\ (continued)}}
%}

\newtheorem{theorem}{Theorem}

\theoremstyle{kjstyle}\newtheorem{thm}{Theorem}
\AtBeginEnvironment{thm}{\begin{tcolorbox}[kjboxstyle]}%
  \AtEndEnvironment{thm}{\end{tcolorbox}}%
\theoremstyle{kjstyle}\newtheorem{lem}[thm]{Lemma}
\AtBeginEnvironment{lem}{\begin{tcolorbox}[kjboxstyle]}%
  \AtEndEnvironment{lem}{\end{tcolorbox}}%
%\theoremstyle{kjstyle}\newtheorem{lem}[thm]{Lemma}
%\AtBeginEnvironment{lem}{\begin{tcolorbox}[kjboxstyle]}%
%  \AtEndEnvironment{lem}{\end{tcolorbox}}%
\theoremstyle{kjstyle}\newtheorem{prop}[thm]{Proposition}
\AtBeginEnvironment{prop}{\begin{tcolorbox}[kjboxstyle]}%
  \AtEndEnvironment{prop}{\end{tcolorbox}}%
\theoremstyle{kjstyle}\newtheorem{cor}[thm]{Corollary}
\AtBeginEnvironment{cor}{\begin{tcolorbox}[kjboxstyle]}%
  \AtEndEnvironment{cor}{\end{tcolorbox}}%

\theoremstyle{kjstyle}\newtheorem{defn}{Definition}
\AtBeginEnvironment{defn}{\begin{tcolorbox}[kjboxstyle]}%
  \AtEndEnvironment{defn}{\end{tcolorbox}}%

\theoremstyle{kjstyle}\newtheorem{ass}{Assumption}
\AtBeginEnvironment{ass}{\begin{tcolorbox}[kjboxstyle]}%
  \AtEndEnvironment{ass}{\end{tcolorbox}}%
\renewcommand\theass{A\arabic{ass}}

\theoremstyle{kjstyle}\newtheorem{conj}{Conjecture}
\AtBeginEnvironment{conj}{\begin{tcolorbox}[kjboxstyle]}%
  \AtEndEnvironment{conj}{\end{tcolorbox}}%


\theoremstyle{kjstyle}\newtheorem{question}{Question}
\AtBeginEnvironment{question}{\begin{tcolorbox}[kjboxstyle]}%
  \AtEndEnvironment{question}{\end{tcolorbox}}%
% \renewcommand\theass{A\arabic{ass}}

\theoremstyle{kjstylenoitalic}\newtheorem{remark}{Remark}
\AtBeginEnvironment{remark}{\begin{tcolorbox}[kjboxstyle]}%
  \AtEndEnvironment{remark}{\end{tcolorbox}}%
\theoremstyle{kjstylenoitalic}\newtheorem{openp}{Open Problem}
\AtBeginEnvironment{openp}{\begin{tcolorbox}[kjboxstyle]}%
  \AtEndEnvironment{openp}{\end{tcolorbox}}%

\theoremstyle{kjstylenoitalic}\newtheorem{ex}{Example}
\AtBeginEnvironment{ex}{\begin{tcolorbox}[kjboxstyle]}%
  \AtEndEnvironment{ex}{\end{tcolorbox}}%


%--------------- other ways to create box
\newcommand{\kjunboxold}[1]{\vspace{.5em}\\\fbox{%
    \begin{minipage}{\linewidth}
      \centering #1
    \end{minipage}%
  }\vspace{.5em}}
\def\lamp{\ensuremath{\lambda_\perp}}
\usepackage{mdframed}
\usepackage{lipsum}
\definecolor{kjgray}{rgb}{.7,.7,.7}
\makeatletter
\newcommand{\kjunbox}[1]{\vspace{.4ex}
  \begin{mdframed}[linecolor=kjgray,innertopmargin=1.3ex,innerleftmargin=.4em,innerrightmargin=.4em,linewidth=1.3pt]
    #1
  \end{mdframed}%
}
\makeatletter
\newcommand{\kjunboxcenter}[1]{\vspace{.4ex}
  \begin{mdframed}[linecolor=kjgray,innertopmargin=1.3ex,innerleftmargin=.4em,innerrightmargin=.4em,linewidth=1.3pt,userdefinedwidth=.8\linewidth,align=center]
    %[linecolor=black,innerleftmargin=.4em,innerrightmargin=.4em,userdefinedwidth=.8\linewidth,align=center]
    #1
  \end{mdframed}%
}
\makeatother


%--- to remove the space before paragraph : change 3.25ex to ...
\makeatletter
\renewcommand{\paragraph}{%
  \@startsection{paragraph}{4}%
  {\z@}{0.50ex \@plus 1ex \@minus .2ex}{-1em}%
  {\normalfont\normalsize\bfseries}%
}
\makeatother

%------- this makes a border that covers other part of the equation....
%- https://tex.stackexchange.com/questions/72692/highlight-an-equation-within-an-align-environment-with-color-option
\usepackage[customcolors,shade]{hf-tikz} % after {}, (right, below) and then (left, above) offset... I don't know how it works; done by trial-and-error.
\newcommand{\hll}[2]{\tikzmarkin[set fill color=yellow!30]{#1}(0.02,-.08)(-0.02,0.3)#2\tikzmarkend{#1}}

%--------------------- better paragraph environment
\newcommand{\myparagraph}[1]{\paragraph{[#1]}\mbox{}\\}

%---- useful sometimes
\def\horizontalline{\noindent\rule{\textwidth}{1pt} }

%--- better table
\usepackage{tabularx} 
\newcolumntype{P}[1]{>{\centering\arraybackslash}p{#1}}
\newcolumntype{M}[1]{>{\centering\arraybackslash}m{#1}}

%%%%%%%%%%%%%%%%%%%%%%%%%%%%%%%%%%%%%%%%%%%%%%%
% symbol definitions
%%%%%%%%%%%%%%%%%%%%%%%%%%%%%%%%%%%%%%%%%%%%%%%

%- hollow
\def\ddefloop#1{\ifx\ddefloop#1\else\ddef{#1}\expandafter\ddefloop\fi}
\def\ddef#1{\expandafter\def\csname #1#1\endcsname{\ensuremath{\mathbb{#1}}}}
\ddefloop ABCDFGHIJKLMNORSTUWXYZ\ddefloop % except for E P V Q

%- c: mathcal
\def\ddef#1{\expandafter\def\csname c#1\endcsname{\ensuremath{\mathcal{#1}}}}
\ddefloop ABCDEFGHIJKLMNOPQRSTUVWXYZ\ddefloop

%- b: mathbf/boldsymbol
\def\ddef#1{\expandafter\def\csname b#1\endcsname{\ensuremath{{\mathbf{#1}}}}}
\ddefloop ABCDEFGHIJKLMNOPQRSTUVWXYZ\ddefloop  
\def\ddef#1{\expandafter\def\csname b#1\endcsname{\ensuremath{{\boldsymbol{#1}}}}}
\ddefloop abcdeghijklmnopqrtsuvwxyz\ddefloop  % except for bf, which often causes an issue

%- h: hat
\def\ddef#1{\expandafter\def\csname h#1\endcsname{\ensuremath{\hat{#1}}}}
\ddefloop ABCDEFGHIJKLMNOPQRSTUVWXYZabcdefghijklmnopqrsuvwxyz\ddefloop % except for 't'
\def\ddef#1{\expandafter\def\csname hc#1\endcsname{\ensuremath{\hat{\mathcal{#1}}}}}
\ddefloop ABCDEFGHIJKLMNOPQRSTUVWXYZ\ddefloop
\def\ddef#1{\expandafter\def\csname hb#1\endcsname{\ensuremath{\hat{\mathbf{#1}}}}}
\ddefloop ABCDEFGHIJKLMNOPQRSTUVWXYZ\ddefloop % 
\def\ddef#1{\expandafter\def\csname hb#1\endcsname{\ensuremath{\hat{\boldsymbol{#1}}}}}
\ddefloop abcdefghijklmnopqrstuvwxyz\ddefloop % 

%- t: til
\def\ddef#1{\expandafter\def\csname t#1\endcsname{\ensuremath{\tilde{#1}}}}
\ddefloop ABCDEFGHIJKLMNOPQRSTUVWXYZabcdefgijklmnpqrtsuvwxyz\ddefloop % except for o and h
\def\ddef#1{\expandafter\def\csname tc#1\endcsname{\ensuremath{\tilde{\mathcal{#1}}}}}
\ddefloop ABCDEFGHIJKLMNOPQRSTUVWXYZ\ddefloop
\def\ddef#1{\expandafter\def\csname tb#1\endcsname{\ensuremath{\tilde{\mathbf{#1}}}}}
\ddefloop ABCDEFGHIJKLMNOPQRSTUVWXYZ\ddefloop
\def\ddef#1{\expandafter\def\csname tb#1\endcsname{\ensuremath{\tilde{\boldsymbol{#1}}}}}
\ddefloop abcdefghijklmnopqrstuvwxyz\ddefloop % 

%- bar
\def\ddef#1{\expandafter\def\csname bar#1\endcsname{\ensuremath{\bar{#1}}}}
\ddefloop ABCDEFGHIJKLMNOPQRSTUVWXYZabcdefghijklmnopqrtsuvwxyz\ddefloop
\def\ddef#1{\expandafter\def\csname barc#1\endcsname{\ensuremath{\bar{\mathcal{#1}}}}}
\ddefloop ABCDEFGHIJKLMNOPQRSTUVWXYZ\ddefloop
\def\ddef#1{\expandafter\def\csname barb#1\endcsname{\ensuremath{\bar{\mathbf{#1}}}}}
\ddefloop ABCDEFGHIJKLMNOPQRSTUVWXYZ\ddefloop
\def\ddef#1{\expandafter\def\csname barb#1\endcsname{\ensuremath{\bar{\boldsymbol{#1}}}}}
\ddefloop abcdefghijklmnopqrstuvwxyz\ddefloop % 

%- war: wide bar
\def\ddef#1{\expandafter\def\csname war#1\endcsname{\ensuremath{\overline{#1}}}}
\ddefloop ABCDEFGHIJKLMNOPQRSTUVWXYZabcdefghijklmnopqrtsuvwxyz\ddefloop
\def\ddef#1{\expandafter\def\csname warc#1\endcsname{\ensuremath{\overline{\mathcal{#1}}}}}
\ddefloop ABCDEFGHIJKLMNOPQRSTUVWXYZ\ddefloop
\def\ddef#1{\expandafter\def\csname warb#1\endcsname{\ensuremath{\overline{\mathbf{#1}}}}}
\ddefloop ABCDEFGHIJKLMNOPQRSTUVWXYZ\ddefloop
\def\ddef#1{\expandafter\def\csname warb#1\endcsname{\ensuremath{\overline{\boldsymbol{#1}}}}}
\ddefloop abcdefghijklmnopqrstuvwxyz\ddefloop % 

%- exceptions
\def\bff{{\boldsymbol f}}
\def\hbff{{\hat{\boldsymbol f}}}
\def\hatt{\hat{t}}
\def\tilo{{\tilde{o}}}
\def\tilh{{\tilde{h}}}
\def\bell{{{\boldsymbol\ell}}}
\def\tell{\ensuremath{\tilde{\ell}}} 
\def\btell{\ensuremath{\widetilde{\boldsymbol{\ell}}}} 
\def\hell{{{\hat\ell}}}

\def\rialpha{\ensuremath{{\mathring{\alpha}}}} 
\def\riz{\ensuremath{\mathring{z}}} 
\def\ribeta{\ensuremath{\mathring{\beta}}} 

%--- generic shortcuts
\newcommand{\fr}[2]{ { \frac{#1}{#2} }}
\newcommand{\tfr}[2]{ { \tfrac{#1}{#2} }}
\newcommand{\til}[1]{{\ensuremath{\tilde{#1}}}}
\newcommand{\wtil}[1]{{\ensuremath{\widetilde{#1}}}}
\newcommand{\wil}[1]{{\ensuremath{\widetilde{#1}}}}
\newcommand{\wbar}[1]{{\ensuremath{\overline{#1}}}}
\newcommand{\war}[1]{{\ensuremath{\overline{#1}}}}
\newcommand{\what}[1]{{\ensuremath{\widehat{#1}}}}
\newcommand{\wat}[1]{{\ensuremath{\widehat{#1}}}}
\newcommand\msf[1]{{\mathsf{#1}}}
\newcommand{\T}{\top}
\def\wed{\wedge}
\def\tsty{\textstyle}
\def\bec{\because}
\newcommand{\nab}{\nabla}
\def\cd{\cdot}
\def\cc{{\circ}}
\def\la{\langle}
\def\ra{\rangle}
\def\dsum{\ensuremath{\displaystyle\sum}} 
\def\der{\ensuremath{\partial}\xspace}
\def\llfl{\left\lfloor} 
\def\rrfl{\right\rfloor}  
\def\llcl{\left\lceil}  
\def\rrcl{\right\rceil}  
\def\lfl{\lfloor} 
\def\rfl{\rfloor}  
\def\lcl{\lceil}  
\def\rcl{\rceil}  
\def\larrow{\ensuremath{\leftarrow}\xspace} 
\def\rarrow{\ensuremath{\rightarrow}\xspace} 
\def\sm{{\ensuremath{\setminus}\xspace} }
\def\grad{\ensuremath{\mathbf{\nabla}}\xspace}  
\def\lt{\left}
\def\rt{\right}

%- checkmarks.
\definecolor{mygrn}{rgb}{0,.8,0}
\definecolor{myred}{rgb}{.8,0,0}
\newcommand{\mycm}{\textcolor{myred}{\cmark}}
\newcommand{\myxm}{\textcolor{myred}{\xmark}}

%--- greek
\def\sig{\sigma}
\def\om{\omega}
\def\dt{\delta}
\def\gam{\gamma}
\def\lam{\lambda}
\def\kap{\kappa}
\def\eps{\varepsilon}
\def\epsilon{\varepsilon}
\def\th{\theta}
\def\Lam{\Lambda}
\def\Dt{\Delta}
\def\Gam{\Gamma}
\def\Sig{\Sigma}
\def\Th{\Theta} 
\def\Om{\Omega}

\usepackage{pgffor}
\def\greeksymbols{alpha,beta,gamma,gam,delta,dt,eps,epsilon,zeta,eta,theta,th,iota,kappa,kap,lambda,lam,mu,nu,xi,pi,rho,sigma,sig,tau,phi,chi,psi,omega,om,Gamma,Gam,Delta,Dt,Theta,Th,Lambda,Lam,Pi,Sigma,Sig,Phi,Psi,Omega,Om}
\def\greeksymbolsnoeta{alpha,beta,gamma,gam,delta,dt,eps,epsilon,zeta,theta,th,iota,kappa,kap,lambda,lam,mu,nu,xi,pi,rho,sigma,sig,tau,phi,chi,psi,omega,om,Gamma,Gam,Delta,Dt,Theta,Th,Lambda,Lam,Pi,Sigma,Sig,Phi,Psi,Omega,Om} % except for eta

%- prefix b: boldsymbol
\foreach \x in \greeksymbolsnoeta{\expandafter\xdef\csname b\x\endcsname{\noexpand\ensuremath{\noexpand\boldsymbol{\csname \x\endcsname}}}}
\def\bfeta{{\boldsymbol \eta}}

%- prefix h, hb: hat, hat bold
\foreach \x in \greeksymbols{\expandafter\xdef\csname h\x\endcsname{\noexpand\ensuremath{\noexpand\hat{\csname \x\endcsname}}}}
\foreach \x in \greeksymbolsnoeta{\expandafter\xdef\csname hb\x\endcsname{\noexpand\ensuremath{\noexpand\hat{\noexpand\boldsymbol{ \csname \x\endcsname}}}}}
\def\hbfeta{{\hat{\boldsymbol \eta}}}

%- prefix bar, barb: bar, bar bold
\foreach \x in \greeksymbols{\expandafter\xdef\csname bar\x\endcsname{\noexpand\ensuremath{\noexpand\bar{\csname \x\endcsname}}}}
\foreach \x in \greeksymbolsnoeta{%
\expandafter\xdef\csname barb\x\endcsname{\noexpand\ensuremath{\noexpand\bar{\noexpand\boldsymbol{ \csname \x\endcsname}}}}
}
\def\barbfeta{{\bar{\boldsymbol \eta}}}

%- prefix t, tb: tilde, tilde bold
\foreach \x in \greeksymbols{\expandafter\xdef\csname t\x\endcsname{\noexpand\ensuremath{\noexpand\tilde{\csname \x\endcsname}}}}
\foreach \x in \greeksymbolsnoeta{\expandafter\xdef\csname tb\x\endcsname{\noexpand\ensuremath{\noexpand\tilde{\noexpand\boldsymbol{ \csname \x\endcsname}}}}}
\def\tbfeta{{\tilde{\boldsymbol \eta}}}

\def\dmu{{\dot\mu}}
\def\ddmu{{\ddot\mu}}

%--- space
\def\RR{{\mathbb{R}}}
\def\NN{{\mathbb{N}}}
\def\ZZ{{\mathbb{Z}}}
\def\SS{{\mathbb{S}}}
\def\HH{{\mathbb{H}}} 

%--- math operators
\DeclareMathOperator{\EE}{\mathbb{E}} % ensures the left space is proper (e.g., 2\EE[X])
\DeclareMathOperator{\VV}{\mathbb{V}}
\DeclareMathOperator{\PP}{\mathbb{P}}
\DeclareMathOperator{\QQ}{\mathbb{Q}}
\DeclareMathOperator{\hEE}{\hat{\mathbb{E}}}
\DeclareMathOperator{\Var}{{\text{Var}}}
\DeclareMathOperator{\barln}{\overline\ln}%

\def\clip#1{\wbar{\del{#1}}}
\DeclareMathOperator{\one}{\mathds{1}\hspace{-.1em}}
\def\onec#1{\one\cbr{#1}}

% \newcommand{\binom}[2]{{\begin{pmatrix} %- already exists in some package.. (maybe commath)
%     #1 \\ #2
% \end{pmatrix}}}              
\newcommand{\normz}[1]{{\norm[0]{#1}}}
\DeclarePairedDelimiterX{\inp}[2]{\langle}{\rangle}{#1, #2}

%-- relationship operators
\newcommand\declareop[3]{%
  \newcommand#1{%
    \mskip\muexpr\medmuskip*#2\relax
    {#3}%
    \mskip\muexpr\medmuskip*#2\relax
}}
\declareop\capprox{1}{{\sr{\const}{\approx}}} % I think 1 means the amount of space.
\declareop\logapprox{1}{{\sr{\mathsf{log}}{\approx}}} % I think 1 means the amount of space.

\newcommand{\lapp}{\mathop{}\!\lessapprox} 
\newcommand{\gapp}{\mathop{}\!\gtrapprox}
\newcommand{\lsim}{\mathop{}\!\lesssim}
\newcommand{\gsim}{\mathop{}\!\gtrsim}

%---- distribution names
\def\Bin{\mathsf{Bin}}
\def\Uniform{{\mathsf{Uniform}}}
\def\Bernoulli{{\ensuremath{\mathsf{Bernoulli}}}}

%--- string
\def\kt{{\mathsf{kt}}}
\def\mle{{\mathsf{mle}}}
\DeclareMathOperator{\Supp}{{\mathsf{Supp}}}
\def\Approx{{\mathsf{Approx}} }
\def\denom{{\mathsf{denom}}}
\def\eff{{\mathsf{eff}}}
\def\Seff{{S_{\mathsf{eff}}}}
\def\opt{{\mathsf{opt}}}
\def\pes{{\mathsf{pes}}}
\def\faury{{\mathsf{faury}}}
\def\nice{{\textsf{nice} } } 
\def\eff{{\mathsf{eff}}}
\def\ErrPrb{{\mathsf{ErrPrb}}}
\def\Seg{{\mathsf{Seg}}}
\def\COM{\mathsf{COM}}
\def\const{\mathsf{const}}
\def\wo{{\ensuremath{\mathsf{wo}}}}
\def\err{\mathsf{err}} 
\def\Top{\mathsf{Top}}
\def\Bot{\mathsf{Bot}}
\def\Sim{\mathsf{Sim}}
\def\TV{\mathsf{TV}}
\def\tmin{{\min}}
\def\tmax{{\max}}
\DeclareMathOperator{\rank}{{\mathsf{rank}}}
\def\kl{{\mathsf{kl}}}
\def\err{\mathsf{err}} 
\def\logloss{{\mathsf{logloss}}}
\def\Ber{{\mathsf{Ber}}}
\def\erf{{\text{erf}}}
\def\erfc{\mathsf{erfc}}
\def\rcF{\ensuremath{\mathring{\cF}}} 
\def\Cf{{\ensuremath{{\normalfont{\text{Cf}}}}}}
\def\barCf{{\ensuremath{\wbar{\text{Cf}}}}}
\newcommand{\Fb}{{\ensuremath{\text{Fb}}}}
\DeclareMathOperator{\SReg}{{\mathsf{SReg}}}
\DeclareMathOperator{\MisPrb}{{{\mathsf{MisPrb}}}}
\def\SR{{\ensuremath{\text{SR}}}\xspace}
\def\lin{{\ensuremath{\mathsf{lin}}}}
\newcommand{\FSB}{{\ensuremath{\text{FSB}}}}
\newcommand{\Cr}{{\ensuremath{\text{Cr}}}}
\newcommand{\SE}{{\ensuremath{\text{SE}}}}
\def\IC{{\ensuremath{\normalfont{\text{IC}}}}}
\newcommand{\Fs}{{\ensuremath{\text{Fs}}}}
\newcommand{\barFs}{{\ensuremath{\wbar{\text{Fs}}}}}
\def\Mis{{\text{Mis}}}
\def\Reward{\ensuremath{\text{Reward}}}
\def\poly{\operatorname{poly}}
\def\Misid{\operatorname{Misid}}
\def\Corral{\ensuremath{\normalfont{\textsc{Corral}}}\xspace}
\def\AUL{{\ensuremath{\normalfont{\text{AUL}}}}} 
\def\Rel{{\ensuremath{\normalfont{\text{Rel}}}}} 
\def\Mis{{\ensuremath{\normalfont{\text{Mis}}}}} 
\def\Rad{\ensuremath{\text{\normalfont{Rad}}}} 
\DeclareMathOperator*{\argmax}{arg~max}
\DeclareMathOperator*{\argmin}{arg~min}
\DeclareMathOperator{\diag}{{\text{diag}}}
\DeclareMathOperator{\supp}{{\text{supp}}}
\DeclareMathOperator{\sign}{{\text{sign}}}
\DeclareMathOperator{\KL}{{\mathsf{KL}}}

\def\Reg{{\mathsf{Reg}}}
\def\Regret{\ensuremath{\normalfont{\text{Regret}}}}
\def\Wealth{\ensuremath{\normalfont{\text{Wealth}}}}
\def\Active{\ensuremath{\text{Active}}}
\def\decomp{\ensuremath{\mbox{decomp}}\xspace}
\def\sym{{\ensuremath{\text{Sym}}\xspace}} 
\def\suchthat{\ensuremath{\text{ s.t. }}} 


%--- min/max
\newcommand{\hyphen}{{\text{-\hspace{-.06em}}}}
\newcommand\kmax[1]{\mathop{#1\hyphen\max}}
\newcommand\kmin[1]{\mathop{#1\hyphen\min}}

%--- colored checkmarks.
\usepackage{pifont}% http://ctan.org/pkg/pifont
\newcommand{\cmark}{\ding{51}}%
\newcommand{\cm}{\ding{51}}%
\newcommand{\gyes}{{\color[rgb]{0,.8,0}\cmark}}
\newcommand{\xmark}{\ding{55}}%
\newcommand{\xm}{\ding{55}}%
\newcommand{\rno}{{\color[rgb]{.8,0,0}\xmark}}
\newcommand{\no}{\mbox{\clr{\ding{55}}}}
\newcommand{\yes}{\mbox{\clg{\ding{51}}}}

%--- add equations above or below.
\newcommand{\ubrace}[2]{{\underbrace{#1}_{\textstyle #2}}}
\newcommand{\sr}[2]{ {\stackrel{#1}{#2}} }

%--- parenthesis or symbols with parenthesis
\newcommand{\evt}[1]{\envert{#1}}
\newcommand*{\medcup}{\mathbin{\scalebox{1.5}{\ensuremath{\cup}}}}%
\newcommand*{\medcap}{\mathbin{\scalebox{1.5}{\ensuremath{\cap}}}}%
\def\bigmid{\,\middle|\,\xspace}
\newcommand{\bmid}{\;\middle|\;}
%- super large parenthesis
\makeatletter
\newcommand{\vast}{\bBigg@{3}}
\newcommand{\Vast}{\bBigg@{4}}
\makeatother

%--- etc.
\def\rhoX{{\rho_{\mathcal{X}}}}
\def\lammin{{\lambda_{\min}}}
\def\elllog{{\ell^{\mathsf{log}}}}




















