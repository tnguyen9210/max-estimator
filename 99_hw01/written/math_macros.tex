
% \DeclareMathOperator*{\err}{err}

% shortcut math symbols
\newcommand{\thetahat}{\hat{\theta}}
\newcommand{\fbo}{f^{\textrm{BO}}}

\newcommand{\efn}{\sqrt{\frac{2}{m}\log\left( \frac{2|H|}{\delta} \right)}}

\newcommand{\hstar}{h^*}
\newcommand{\hhat}{\hat{h}}
\newcommand{\hhatB}{\hat{h}^{\mathcal{B}}}
\newcommand{\hhatA}{\hat{h}^{\mathcal{A}}}

\newcommand{\wstar}{w^*}
\newcommand{\what}{\hat{w}}

\newcommand{\EEhat}{\widehat{\mathbb{E}}}
\newcommand{\Lhat}{\widehat{L}}


\newcommand{\lsq}{l_{\textrm{sq}}}
\newcommand{\yhat}{\hat{y}}
\newcommand{\rad}{\textrm{Rad}}
\newcommand{\radhat}{\widehat{\textrm{Rad}}}
\newcommand{\eps}{\sigma}
\newcommand{\acti}{\phi}

\newcommand{\idt}[1]{\mathbbm{1}\{#1\}}
\newcommand{\mcmt}[1]{\text{(#1)}}


\newcommand{\EE}{\mathbb{E}}
\newcommand{\PP}{\mathbb{P}}

\newcommand{\RR}{\mathbb{R}}
\newcommand{\XX}{\bm{\mathcal{X}}}

\newcommand{\CC}{\mathbb{C}}

\newcommand{\Hcal}{\mathcal{H}}
\newcommand{\Acal}{\mathcal{A}}
\newcommand{\Bcal}{\mathcal{B}}
\newcommand{\Gcal}{\mathcal{G}}
\newcommand{\Fcal}{\mathcal{F}}
\newcommand{\Xcal}{\mathcal{X}}
\newcommand{\Ycal}{\mathcal{Y}}
\newcommand{\Wcal}{\mathcal{W}}
\newcommand{\Scal}{\mathcal{S}}
% \newcommand{\hHcal}{\hat{\mathcal{H}}}

\providecommand{\mat}[1]{\bm{\mathrm{#1}}}%
\renewcommand{\vec}[1]{\bm{\mathrm{#1}}}
% \providecommand{\mat}[1]{\bm{#1}}%
% \renewcommand{\vec}[1]{\bm{#1}}

\providecommand{\eye}{\mat{I}}
\providecommand{\mA}{\ensuremath{\mat{A}}}
\providecommand{\mB}{\ensuremath{\mat{B}}}
\providecommand{\mC}{\ensuremath{\mat{C}}}
\providecommand{\mD}{\ensuremath{\mat{D}}}
\providecommand{\mE}{\ensuremath{\mat{E}}}
\providecommand{\mF}{\ensuremath{\mat{F}}}
\providecommand{\mG}{\ensuremath{\mat{G}}}
\providecommand{\mH}{\ensuremath{\mat{H}}}
\providecommand{\mI}{\ensuremath{\mat{I}}}
\providecommand{\mJ}{\ensuremath{\mat{J}}}
\providecommand{\mK}{\ensuremath{\mat{K}}}
\providecommand{\mL}{\ensuremath{\mat{L}}}
\providecommand{\mM}{\ensuremath{\mat{M}}}
\providecommand{\mN}{\ensuremath{\mat{N}}}
\providecommand{\mO}{\ensuremath{\mat{O}}}
\providecommand{\mP}{\ensuremath{\mat{P}}}
\providecommand{\mQ}{\ensuremath{\mat{Q}}}
\providecommand{\mR}{\ensuremath{\mat{R}}}
\providecommand{\mS}{\ensuremath{\mat{S}}}
\providecommand{\mT}{\ensuremath{\mat{T}}}
\providecommand{\mU}{\ensuremath{\mat{U}}}
\providecommand{\mV}{\ensuremath{\mat{V}}}
\providecommand{\mW}{\ensuremath{\mat{W}}}
\providecommand{\mX}{\ensuremath{\mat{X}}}
\providecommand{\mY}{\ensuremath{\mat{Y}}}
\providecommand{\mZ}{\ensuremath{\mat{Z}}}


\providecommand{\mzero}{\ensuremath{\mat{0}}}
\providecommand{\mSigma}{\ensuremath{\mat{\Sigma}}}
\providecommand{\mLambda}{\ensuremath{\mat{\Lambda}}}

\providecommand{\ones}{\vec{e}}
\providecommand{\va}{\ensuremath{\vec{a}}}
\providecommand{\vb}{\ensuremath{\vec{b}}}
\providecommand{\vc}{\ensuremath{\vec{c}}}
\providecommand{\vd}{\ensuremath{\vec{d}}}
\providecommand{\ve}{\ensuremath{\vec{e}}}
\providecommand{\vf}{\ensuremath{\vec{f}}}
\providecommand{\vg}{\ensuremath{\vec{g}}}
\providecommand{\vh}{\ensuremath{\vec{h}}}
\providecommand{\vi}{\ensuremath{\vec{i}}}
\providecommand{\vj}{\ensuremath{\vec{j}}}
\providecommand{\vk}{\ensuremath{\vec{k}}}
\providecommand{\vl}{\ensuremath{\vec{l}}}
\providecommand{\vm}{\ensuremath{\vec{l}}}
\providecommand{\vn}{\ensuremath{\vec{n}}}
\providecommand{\vo}{\ensuremath{\vec{o}}}
\providecommand{\vp}{\ensuremath{\vec{p}}}
\providecommand{\vq}{\ensuremath{\vec{q}}}
\providecommand{\vr}{\ensuremath{\vec{r}}}
\providecommand{\vs}{\ensuremath{\vec{s}}}
\providecommand{\vt}{\ensuremath{\vec{t}}}
\providecommand{\vu}{\ensuremath{\vec{u}}}
\providecommand{\vv}{\ensuremath{\vec{v}}}
\providecommand{\vw}{\ensuremath{\vec{w}}}
\providecommand{\vx}{\ensuremath{\vec{x}}}
\providecommand{\vy}{\ensuremath{\vec{y}}}
\providecommand{\vz}{\ensuremath{\vec{z}}}

\providecommand{\vpi}{\ensuremath{\vecalt{\pi}}} 
\providecommand{\vmu}{\ensuremath{\vec{\mu}}}
\providecommand{\veps}{\ensuremath{\vec{\epsilon}}} 

\providecommand{\abs}[1]{\left| #1 \right|}
% \providecommand{\norm}[1]{\lVert#1\rVert}
\providecommand{\norm}[1]{\left\|#1\right\|}

% operator argmax and argmin
\DeclareMathOperator*{\argmax}{arg\,max}
\DeclareMathOperator*{\argmin}{arg\,min}

\newcommand{\var}{\textrm{Var}}
\newcommand{\pluseq}{\mathrel{+}=}
\newcommand{\asteq}{\mathrel{*}=}
\newcommand{\deq}{\mathrel{:}=}
% comment
\newcommand{\cmmt}[1]{\{\textit{#1}\}}
% \newcommand{\cmmt}[1]{\ignorespaces}

\makeatletter
\newcommand*{\indep}{%
  \mathbin{%
    \mathpalette{\@indep}{}%
  }%
}
\newcommand*{\nindep}{%
  \mathbin{%                   % The final symbol is a binary math operator
    \mathpalette{\@indep}{\not}% \mathpalette helps for the adaptation
                               % of the symbol to the different math styles.
  }%
}
\newcommand*{\@indep}[2]{%
  % #1: math style
  % #2: empty or \not
  \sbox0{$#1\perp\m@th$}%        box 0 contains \perp symbol
  \sbox2{$#1=$}%                 box 2 for the height of =
  \sbox4{$#1\vcenter{}$}%        box 4 for the height of the math axis
  \rlap{\copy0}%                 first \perp
  \dimen@=\dimexpr\ht2-\ht4-.2pt\relax
      % The equals symbol is centered around the math axis.
      % The following equations are used to calculate the
      % right shift of the second \perp:
      % [1] ht(equals) - ht(math_axis) = line_width + 0.5 gap
      % [2] right_shift(second_perp) = line_width + gap
      % The line width is approximated by the default line width of 0.4pt
  \kern\dimen@
  {#2}%
      % {\not} in case of \nindep;
      % the braces convert the relational symbol \not to an ordinary
      % math object without additional horizontal spacing.
  \kern\dimen@
  \copy0 %                       second \perp
} 
\makeatother


