
\section*{Q1}
\newpage
a) The loopy BP converges after 1 iteration.

Explanation:

Assuming we initialize messages from variables to factors are uniform, e.g. $m_{A \to h}(A)=<0.5,0.5>$

We then compute the messages from factors to variables. Since the factors and messages are the same and the structure is symmetric, we only need to compute one message, e.g. $m_{h \to C}(C)$, other messages will be the same.

We compute $m_{h \to C}(C)$
\begin{align*}
  m_{h \to C}(C) = \sum_{a}h(a,C)m_{A \to h}(a) = <2.5,2.5>
\end{align*}

After renormalizing this message, we have $m_{h \to C}(C) = <0.5,0.5>$. Other messages from factors to variables will be the same.

We then compute messages from variables to factors. Since the factors and messages are the same and the structure is symmetric, we only need to compute one message, e.g., $m_{A \to h}(A)=<0.5,0.5>$, other messages will be the same.

We compute $m_{A \to h}(A)$
\begin{align*}
  m_{A \to h}(A) = m_{f \to A}(A) = <0.5,0.5>
\end{align*}

This message $m_{A \to h}(A)$ converged. Since the factors and messages are the same and the structure is symmetric, other messages will be converged as well. Hence the loopy BP converged after first iteration.

\pagebreak
b) We only need to compute one marginal $P(A)$, the other marginals $P(B)$, $P(C)$ will be the same.
\begin{align*}
  P(A) = m_{f \to A}(A)*m_{h \to A}(A) = <0.5, 0.5>
\end{align*}

We only need to compute one marginal $P(A,B)$, the other marginals $P(B,C)$, $P(C,A)$ will be the same.
\begin{align*}
  P(A,B) = f(A,B)*m_{A \to f}(A)*m_{B \to f}(B) = <0.1,0.4,0.4,0.1>
\end{align*}

\begin{tabular}{|c|c|}
  \hline
  $A,B$  & $P(A,B)$ \\
  \hline
  $a^0 b^0$ & 0.1 \\
  $a^0 b^1$ & 0.4 \\
  $a^1 b^0$ & 0.4 \\
  $a^1 b^1$ & 0.1 \\
  \hline
\end{tabular}

c) They are correct 

\pagebreak

\section*{Q2}

a)

\begin{tabular}{|c|c|c|c|}
  \hline
    & $S1$ & $S2$ & $S3$ \\
  \hline
  $S1$ & 0.9 & 0.1 & 0.0 \\
  \hline
  $S2$ & 0.0 & 0.9 & 0.1 \\
  \hline
  $S3$ & 0.1 & 0.0 & 0.9 \\
  \hline
\end{tabular}

b) To compute the stationary distributions we solve this system of equations
\begin{align*}
  & \pi(S1) = 0.9\pi(S1) + 0.1\pi(S3)  \\
  & \pi(S2) = 0.9\pi(S2) + 0.1\pi(S1) \\
  & \pi(S3) = 0.9\pi(S3) + 0.1\pi(S2)  \\
  & \pi(S3) = 0.9\pi(S3) + 0.1\pi(S2)  \\
  & \pi(S1) + \pi(S2) + \pi(S3) = 1 
\end{align*}
The solution of the stationary distributions is $\pi = [1/3,1/3,1/3]$

c) This stationary distribution does not satisfy detailed balance
\begin{align*}
  \pi(S1)*T(S1 \to S2) \ne \pi(S2)*T(S2 \to S1)
\end{align*}
\begin{align*}
  &\pi(S1)*T(S1 \to S2) = 1/3*0.1 = 1/30 \\
  &\pi(S2)*T(S2 \to S1) = 1/3*0.0 = 0.0 \\
\end{align*}


\pagebreak