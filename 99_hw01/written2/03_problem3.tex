

\newpage
\subsection*{Problem 3.}

\subsubsection*{Problem 3.a.}

By definition, we have
\begin{align*}
  Q =  (1-2\eps)\begin{pmatrix}
    1 & -\tau \\
    -\tau & \tau^2
  \end{pmatrix}
            + \eps \begin{pmatrix}
              1 & \tau \\
              \tau & \tau^2
            \end{pmatrix}
                     + \eps \begin{pmatrix}
                       0 & 0 \\
                       0 & 1
                     \end{pmatrix}
\end{align*}

We set $\eps = \tau^2$, then
\begin{align*}
  Q &=  (1-2\tau^2)\begin{pmatrix}
    1 & -\tau \\
    -\tau & \tau^2
  \end{pmatrix}
            + \tau^2 \begin{pmatrix}
              1 & \tau \\
              \tau & \tau^2
            \end{pmatrix}
                     + \tau^2 \begin{pmatrix}
                       0 & 0 \\
                       0 & 1
                     \end{pmatrix} \\
    &= \begin{pmatrix}
      1-\tau^2 & 3\tau^3 - r \\
      3\tau^3 - r & -\tau^4 + 2\tau^2
    \end{pmatrix}
\end{align*}

Use wolfram alpha to find the $\Qinv$, we have
\begin{align*}
  \Qinv = \frac{1}{\tau^2(8\tau^4 - 3\tau^2 -1)} \begin{pmatrix}
    \tau^2(\tau^2 - 2) & r(3\tau^3 - 1) \\
    r(3\tau^3 - 1) & \tau^2 -  1
  \end{pmatrix}
\end{align*}

\subsubsection*{Problem 3.b.}

From $\Qinv$, we have $(\Qinv)_{11} = \frac{\tau^4 - 2\tau^2}{\tau^2(8\tau^4 - 3\tau^2 -1)} $

Compute $R_1$, $R_2$, and $R_3$
\begin{align*}
  R_1 &= \abs{e_1^T\Qinv x^{(1)}} \\
      &= \abs{\frac{-2\tau^4 - \tau^2}{\tau^2(8\tau^4 - 3\tau^2 -1)}} \\
      &= \abs{\frac{2\tau^4 + \tau^2}{\tau^2(8\tau^4 - 3\tau^2 -1)}}
\end{align*}

\begin{align*}
  R_2 &= \abs{e_1^T\Qinv x^{(2)}} \\
      &= \abs{\frac{4\tau^4 - 3\tau^2}{\tau^2(8\tau^4 - 3\tau^2 -1)}} 
\end{align*}

\begin{align*}
  R_3 &= \abs{e_1^T\Qinv x^{(3)}} \\
      &= \abs{\frac{3\tau^3 - \tau}{\tau^2(8\tau^4 - 3\tau^2 -1)}} 
\end{align*}

Because $\tau \le 1$, the term order of $\tau$ will dominate other term, hence orderwise, $R_3$ is the largest range. And orderwise, $R_3$ can be much larger than $(\Qinv)_{11}$. So $i^* = 3$

Compute the $\frac{R_3}{(\Qinv)_{11}}$
\begin{align*}
  \frac{R_3}{(\Qinv)_{11}} & = \abs{\frac{3\tau^3 - \tau}{\tau^4 - 2\tau^2}} = O\del{\frac{1}{\tau}}
\end{align*}

So the ratio $\frac{R_3}{(\Qinv)_{11}}$ scales with $O\del{\frac{1}{\tau}}$ with $\tau < 1$. The case shows that the range $R_3$ can be much larger $(\Qinv)_{11}$, and hence the Catoni estimator step of PopArt is necessary to perform well in the nonasymptotic regime.